
\newcommand{\cxueke}{工学}
\newcommand{\exueke}{Engineering}
\newcommand{\cxuewei}{硕士}
\newcommand{\cxueweishort}{硕}
\newcommand{\exuewei}{Master}
\newcommand{\exueweier}{Master's}

\newcommand{\yihao}{\fontsize{26pt}{26pt}\selectfont}
\newcommand{\xiaoyi}{\fontsize{24pt}{24pt}\selectfont}
\newcommand{\erhao}{\fontsize{22pt}{1.25\baselineskip}\selectfont}
\newcommand{\xiaoer}{\fontsize{18pt}{18pt}\selectfont}
\newcommand{\sanhao}{\fontsize{16pt}{16pt}\selectfont}
\newcommand{\xiaosan}{\fontsize{15pt}{15pt}\selectfont}
\newcommand{\sihao}{\fontsize{14pt}{14pt}\selectfont}
\newcommand{\xiaosi}{\fontsize{12pt}{12pt}\selectfont}
\newcommand{\wuhao}{\fontsize{10.5pt}{10.5pt}\selectfont}
\newcommand{\xiaowu}{\fontsize{9pt}{9pt}\selectfont}

% pdf书签只能包含普通字符
\newcommand{\myquad}{\texorpdfstring{\quad}{~~}}

\makeatletter
\gdef\hitempty{}

% 使图编号为 7-1 的格式 %\protect{~}
\renewcommand{\thefigure}{\arabic{chapter}-\arabic{figure}}
% 使子图编号为 a)的格式
\renewcommand{\thesubfigure}{\alph{subfigure})}
% 使子图引用为 7-1 a) 的格式
% 母图编号和子图编号之间用~加一个空格
\renewcommand{\p@subfigure}{\thefigure~}
% 使表编号为 7-1 的格式
\renewcommand{\thetable}{\arabic{chapter}-\arabic{table}}
% 使公式编号为 7-1 的格式
\renewcommand{\theequation}{\arabic{chapter}-\arabic{equation}}

\newcommand{\algoenname}{Algo.} %算法英文标题
\newfloatlist[chapter]{algoen}{aen}{\listalgoenname}{\algoenname}
\newfixedcaption{\algoencaption}{algoen}
\renewcommand{\thealgoen}{\thechapter-\arabic{algocf}}
\renewcommand{\@cftmakeaentitle}{\chapter*{\listalgoenname\@mkboth{\bfseries\listalgoenname}{\bfseries\listalgoenname}}}

\renewcommand{\algorithmcfname}{算法}
\setlength\AlCapSkip{1.2ex}
\SetAlgoSkip{1pt}
\renewcommand{\algocf@captiontext}[2]{\wuhao#1\algocf@typo ~ \AlCapFnt{}#2} % text of caption
\expandafter\ifx\csname algocf@within\endcsname\relax% if \algocf@within doesn't exist
\renewcommand\thealgocf{\@arabic\c@algocf} % and the way it is printed
\else%                                    else
\renewcommand\thealgocf{\csname the\algocf@within\endcsname-\@arabic\c@algocf}
\fi
\renewcommand{\algocf@makecaption}[2]{%中英文双标题一定多于一行,因此去掉单行时的判断,并将\parbox中标题设置为居中
  \addtolength{\hsize}{\algomargin}%
  \sbox\@tempboxa{\algocf@captiontext{#1}{#2}}%
    \hskip .5\algomargin%
    \parbox[t]{\hsize}{\centering\algocf@captiontext{#1}{#2}}% 
  \addtolength{\hsize}{-\algomargin}%
}
\newcommand{\AlgoBiCaption}[2]{%直接取出自定义的中英文标题条目加入到真正的\caption 中  
   \caption[#1]{\protect\setlength{\baselineskip}{1.5em}#1 \protect \\ Algo. \thealgocf~~ #2} % \algoencaption{#2}   
   \addcontentsline{aen}{algoen}{\protect\numberline{\thealgoen}{#2}}
   }

\setlength{\algoheightruledefault}{1.5pt}
\newcommand{\foocaption}[1]{ \def\@algocf@pre@plainruled{\hrule height1.5pt depth0pt\kern\interspacetitleruled #1 \kern\interspacealgoruled\hrule height1pt depth0pt\kern\interspacetitleruled} }
\def\@algocf@post@ruled{\kern\interspacealgoruled\hrule height1.5pt\relax}%
\makeatother
