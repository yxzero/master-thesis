
\newcommand{\cxueke}{工学}
\newcommand{\exueke}{Engineering}
\newcommand{\cxuewei}{硕士}
\newcommand{\cxueweishort}{硕}
\newcommand{\exuewei}{Master}
\newcommand{\exueweier}{Master's}

\newcommand{\yihao}{\fontsize{26pt}{26pt}\selectfont}
\newcommand{\xiaoyi}{\fontsize{24pt}{24pt}\selectfont}
\newcommand{\erhao}{\fontsize{22pt}{1.25\baselineskip}\selectfont}
\newcommand{\xiaoer}{\fontsize{18pt}{18pt}\selectfont}
\newcommand{\sanhao}{\fontsize{16pt}{16pt}\selectfont}
\newcommand{\xiaosan}{\fontsize{15pt}{15pt}\selectfont}
\newcommand{\sihao}{\fontsize{14pt}{14pt}\selectfont}
\newcommand{\xiaosi}{\fontsize{12pt}{12pt}\selectfont}
\newcommand{\wuhao}{\fontsize{10.5pt}{10.5pt}\selectfont}
\newcommand{\xiaowu}{\fontsize{9pt}{9pt}\selectfont}

% pdf书签只能包含普通字符
\newcommand{\myquad}{\texorpdfstring{\quad}{~~}}

% 研究生院只对图、表、公式的编号做了具体要求
% 使图编号为 7-1 的格式
\renewcommand{\thefigure}{\arabic{chapter}-\arabic{figure}}
% 使子图编号为 a)的格式
\renewcommand{\thesubfigure}{\alph{subfigure})}
% 使子图引用为 7-1 a) 的格式
\makeatletter
\renewcommand{\p@subfigure}{\thefigure~}
\makeatother
% 使表编号为 7-1 的格式
\renewcommand{\thetable}{\arabic{chapter}-\arabic{table}}
% 使公式编号为 7-1 的格式
\renewcommand{\theequation}{\arabic{chapter}-\arabic{equation}}
% Algorithm -> 算法
\renewcommand{\algorithmcfname}{算法}
