% All the packages.

% Page Layout
\usepackage[
a4paper,
text={150true mm, 224true mm},
top=35.5true mm,
left=30true mm,
head=5true mm,
headsep=2.5true mm,
foot=8.5true mm,
]{geometry}
\usepackage{titlesec}
\usepackage{titletoc}
\usepackage{fancyhdr}

% Font
\usepackage{xeCJK}
\setCJKfamilyfont{songticu}{Songti SC Bold}
\newcommand{\songticu}{\CJKfamily{songticu}}
\setmainfont{Times New Roman}
\setsansfont{Arial}
\usepackage{fontspec}

% Figures
\usepackage{graphicx}
\usepackage{tikz}
\usetikzlibrary{positioning}
\usepackage{pgfplots}
\pgfplotsset{compat=1.13}
\usepgfplotslibrary{groupplots}

% Tables
\usepackage{tabularx}

\usepackage{color}          % 支持彩色
\usepackage{amsmath}        % AMSLaTeX宏包 用来排出更加漂亮的公式
\usepackage{amssymb}
\usepackage[below]{placeins}%允许上一个section的浮动图形出现在下一个section的开始部分,还提供\FloatBarrier命令,使所有未处理的浮动图形立即被处理
\usepackage{flafter}       % 使得所有浮动体不能被放置在其浮动环境之前,以免浮动体在引述它的文本之前出现.
\usepackage{multirow}       %使用Multirow宏包,使得表格可以合并多个row格
\usepackage{booktabs}       % 表格,横的粗线;\specialrule{1pt}{0pt}{0pt}
\usepackage{longtable}      %支持跨页的表格。
\usepackage[hang]{subfigure}%支持子图 %centerlast 设置最后一行是否居中
\usepackage[subfigure]{ccaption} %支持双语标题
\usepackage[sort&compress,numbers]{natbib}% 支持引用缩写的宏包
\usepackage{enumitem}       %使用enumitem宏包,改变列表项的格式
\usepackage{calc}           %长度可以用+ - * / 进行计算

% 定理类环境宏包,其中 amsmath 选项用来兼容 AMS LaTeX 的宏包
\usepackage[amsmath,thmmarks,hyperref]{ntheorem}

\usepackage[
xetex,
bookmarksnumbered=true,
bookmarksopen=true,
colorlinks=false,
pdfborder={0 0 1},
citecolor=blue,
linkcolor=red,
anchorcolor=green,
urlcolor=blue,
breaklinks=true,
naturalnames  %与algorithm2e宏包协调
]{hyperref}

% 算法的宏包,注意宏包兼容性,先后顺序为float、hyperref、algorithm(2e),否则无法生成算法列表
\usepackage[plainruled,linesnumbered,algochapter]{algorithm2e}
