
\newcommand{\yihao}{\fontsize{26pt}{26pt}\selectfont}
\newcommand{\xiaoyi}{\fontsize{24pt}{24pt}\selectfont}
\newcommand{\erhao}{\fontsize{22pt}{1.25\baselineskip}\selectfont}
\newcommand{\xiaoer}{\fontsize{18pt}{18pt}\selectfont}
\newcommand{\sanhao}{\fontsize{16pt}{16pt}\selectfont}
\newcommand{\xiaosan}{\fontsize{15pt}{15pt}\selectfont}
\newcommand{\sihao}{\fontsize{14pt}{14pt}\selectfont}
\newcommand{\xiaosi}{\fontsize{12pt}{12pt}\selectfont}
\newcommand{\wuhao}{\fontsize{10.5pt}{10.5pt}\selectfont}
\newcommand{\xiaowu}{\fontsize{9pt}{9pt}\selectfont}

% 研究生院只对图、表、公式的编号做了具体要求
% 使图编号为 7-1 的格式
\renewcommand{\thefigure}{\arabic{chapter}-\arabic{figure}}
% 使表编号为 7-1 的格式
\renewcommand{\thetable}{\arabic{chapter}-\arabic{table}}
% 使公式编号为 7-1 的格式
\renewcommand{\theequation}{\arabic{chapter}-\arabic{equation}}
% Algorithm -> 算法
\renewcommand{\algorithmcfname}{算法}

\theoremstyle{plain}
\theorembodyfont{\songti\rmfamily}
\theoremheaderfont{\heiti\rmfamily}
\newtheorem{definition}{\heiti 定义}[chapter]
\newtheorem{example}{\heiti 例}[chapter]

\allowdisplaybreaks[4]

\setlength{\parindent}{2em}

\arraycolsep=1.6pt

\CTEXsetup[number={\arabic{chapter}}]{chapter}
\renewcommand\chaptername{第~\thechapter~章}

\setcounter{secnumdepth}{4}
\setcounter{tocdepth}{2}

\titleformat{\chapter}{\center\xiaoer\heiti}{\chaptertitlename}{0.5em}{}
\titlespacing{\chapter}{0pt}{-5.5mm}{8mm}
\titleformat{\section}{\xiaosan\heiti}{\thesection}{0.5em}{}
\titlespacing{\section}{0pt}{4.5mm}{4.5mm}
\titleformat{\subsection}{\sihao\heiti}{\thesubsection}{0.5em}{}
\titlespacing{\subsection}{0pt}{4mm}{4mm}
\titleformat{\subsubsection}{\xiaosi\heiti}{\thesubsubsection}{0.5em}{}
\titlespacing{\subsubsection}{0pt}{0pt}{0pt}

\titlecontents{chapter}[3.8em]
{\hspace{-3.8em}\heiti}{\thecontentslabel~~}{}{\titlerule*[4pt]{.}\contentspage}
\dottedcontents{section}[32pt]{}{21pt}{0.3pc}
\dottedcontents{subsection}[53pt]{}{30pt}{0.3pc}


% 按工大标准, 缩小目录中各级标题之间的缩进,使它们相隔一个字符距离,也就是12pt
\makeatletter
\renewcommand*\l@chapter{\@dottedtocline{0}{0em}{5em}}
\renewcommand*\l@section{\@dottedtocline{1}{1em}{1.8em}}
\renewcommand*\l@subsection{\@dottedtocline{2}{2em}{2.5em}}

% 定义页眉和页脚
\newcommand{\makeheadrule}{
\rule[7pt]{\textwidth}{0.75pt} \\[-23pt]
\rule{\textwidth}{2.25pt}}
\renewcommand{\headrule}{
{\if@fancyplain\let\headrulewidth\plainheadrulewidth\fi
\makeheadrule}
}
\pagestyle{fancyplain}

% 去掉章节标题中的数字
% 不要注销这一行,否则页眉会变成:“第1章1  绪论”样式
\renewcommand{\chaptermark}[1]{\markboth{\chaptertitlename~\ #1}{}}
\fancyhf{}

% 在book文件类别下,\leftmark自动存录各章之章名,\rightmark记录节标题
% 页眉字号 工大要求 小五
% 根据单双面打印设置不同的页眉;
\fancyhead[CO]{\xiaowu \songti 哈尔滨工业大学工学硕士学位论文}
\fancyhead[CE]{\xiaowu \songti 哈尔滨工业大学工学硕士学位论文}
\fancyfoot[C,C]{\xiaowu -~\thepage~-}

\renewcommand{\frontmatter}{
\cleardoublepage
\@mainmatterfalse
\pagenumbering{Roman}
}

% 调整罗列环境的布局
\setitemize{leftmargin=0em,itemsep=0em,partopsep=0em,parsep=0em,topsep=0em,itemindent=3em}
\setenumerate{leftmargin=0em,itemsep=0em,partopsep=0em,parsep=0em,topsep=0em,itemindent=3.5em}

\newcommand{\citeup}[1]{\textsuperscript{\cite{#1}}}

% 定制浮动图形和表格标题样式
\captionnamefont{\wuhao}
\captiontitlefont{\wuhao}
\captiondelim{~~}
%\captionstyle{\hang}
\hangcaption
\renewcommand{\subcapsize}{\wuhao}
\setlength{\abovecaptionskip}{0pt}
\setlength{\belowcaptionskip}{0pt}

% 自定义项目列表标签及格式 \begin{publist} 列表项 \end{publist}
\newcounter{pubctr} %自定义新计数器
\newenvironment{publist}{%%%%%定义新环境
\begin{list}{[\arabic{pubctr}]} %%标签格式
    {
     \usecounter{pubctr}
     \setlength{\leftmargin}{1.7em}     % 左边界 \leftmargin =\itemindent + \labelwidth + \labelsep
     \setlength{\itemindent}{0em}     % 标号缩进量
     \setlength{\labelsep}{0.5em}       % 标号和列表项之间的距离,默认0.5em
     \setlength{\rightmargin}{0em}    % 右边界
     \setlength{\topsep}{0ex}         % 列表到上下文的垂直距离
     \setlength{\parsep}{0ex}         % 段落间距
     \setlength{\itemsep}{0ex}        % 标签间距
     \setlength{\listparindent}{0pt} % 段落缩进量
    }}
{\end{list}}%%%%%

% 默认字体
\renewcommand{\normalsize}{
\@setfontsize \normalsize{12pt}{12pt}
\setlength \abovedisplayskip{4pt}
\setlength \abovedisplayshortskip{4pt}
\setlength \belowdisplayskip{\abovedisplayskip}
\setlength \belowdisplayshortskip{\abovedisplayshortskip}
\let \@listi \@listI
}
  
% 设置行距和段落间垂直距离
\newcommand{\defaultfont}{
\renewcommand{\baselinestretch}{1.62}
\normalsize \selectfont
}
% 加大字间距,使每行34个字,若要使得每行33个字,则将0.56pt替换为0.96pt。
\renewcommand{\CJKglue}{\hskip 0.56pt plus 0.08\baselineskip} 
% 公式之前可以换页,公式出现在页面顶部
\predisplaypenalty=0

% 定义封面
\def \makecover{
\begin{titlepage}

% 封面一
\vspace*{0.8cm}
\begin{center}
{\xiaoyi \songticu 工学硕士学位论文}
\vspace{1cm}

\parbox[t][2.8cm][t]{\textwidth}{
\begin{center}{\erhao \heiti \ctitle}\end{center}
}
\parbox[t][5.1cm][t]{\textwidth}{
\begin{center}{\erhao \textbf{\etitle}}\end{center}
}
\parbox[t][7.4cm][t]{\textwidth}{
\begin{center}{\xiaoer \songticu \cauthor}\end{center}
}
\parbox[t][1.4cm][t]{\textwidth}{
\begin{center}{\xiaoer \kaishu \textbf{哈尔滨工业大学}}\end{center}
}
{\xiaoer \songticu \textbf{\cdate}}
\end{center}

% 内封
\newpage
\thispagestyle{empty}

\begin{center}
{\xiaosi \songti
\begin{tabular}{@{}r@{:}l@{}}
国内图书分类号 & \classifiedindex \\
国际图书分类号 & \udc
\end{tabular}
}\hfill
{\xiaosi \songti
\begin{tabular}{@{}r@{:}l@{}}
学校代码 & 10213 \\
密级 & \confidentiality
\end{tabular}
}

\parbox[t][3.2cm][t]{\textwidth}{
\begin{center} \end{center}
}
\parbox[t][2.4cm][t]{\textwidth}{
\begin{center}{\xiaoer \songticu 工学硕士学位论文}\end{center}
}
\parbox[t][5cm][t]{\textwidth}{
\begin{center}{\erhao \heiti \ctitle}\end{center}
}
\parbox[t][9.8cm][b]{\textwidth}{
\begin{center}
\renewcommand{\arraystretch}{1.62}
\sihao \songti
\begin{tabular}{l@{:}l}
{\heiti 硕\hfill 士\hfill 研\hfill 究\hfill 生} & \cauthor \\
{\heiti 导\hfill 师} & \csupervisor \\
{\heiti 申\hfill 请\hfill 学\hfill 位} & \cdegree \\
{\heiti 学\hfill 科} & \csubject \\
{\heiti 所\hfill 在\hfill 单\hfill 位} & \caffiliate \\
{\heiti 答\hfill 辩\hfill 日\hfill 期} & \cdate \\
{\heiti 授予学位单位} & 哈尔滨工业大学
\end{tabular}
\renewcommand{\arraystretch}{1}
\end{center}
}
\end{center}

% 英文封面
\newpage
\thispagestyle{empty}

{\noindent \xiaosi
Classified Index: \classifiedindex \\
U.D.C: \udc
}

\begin{center}
\parbox[t][1.6cm][t]{\textwidth}{
\begin{center} \end{center}
}
\parbox[t][3.5cm][t]{\textwidth}{\xiaoer
\begin{center}{Dissertation for the Master's Degree in Engineering}\end{center}
}
\parbox[t][7cm][t]{\textwidth}{
\begin{center}{\erhao \textbf{\etitle}}\end{center}
}

{\renewcommand{\arraystretch}{1.3}
\sihao
\begin{tabular}{@{}l@{~}l@{}}
\textbf{Candidate:}                     &  \eauthor \\
\textbf{Supervisor:}                    &  \esupervisor \\
\textbf{Academic Degree Applied for:}   &  \edegree \\
\textbf{Specialty:}                     &  \esubject \\
\textbf{Affiliation:}                   &  \eaffiliate \\
\textbf{Date of Defence:}               &  \edate \\
\textbf{Degree-Conferring-Institution:} &  Harbin Institute of Technology
\end{tabular}
\renewcommand{\arraystretch}{1}}

\end{center}
\end{titlepage}


% 摘要
\clearpage

\chapter*{摘\quad 要}
\phantomsection
\addcontentsline{toc}{chapter}{摘要}

\setcounter{page}{1}
\songti \defaultfont
\cabstract
\vspace{\baselineskip}

\hangafter=1\hangindent=51pt\noindent
{\heiti 关键词}:\ckeywords

% Abstract
\clearpage

\chapter*{\textbf{Abstract}}
\phantomsection
\addcontentsline{toc}{chapter}{Abstract}
\addtocontents{toc}{\vspace{\baselineskip}}

\eabstract
\vspace{\baselineskip}

\hangafter=1\hangindent=60pt\noindent
{\textbf{Keywords:}}  \ekeywords

% End make cover.
}
