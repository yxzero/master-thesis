%!TEX root = ../main.tex

\chapter*{结\quad 论}
\phantomsection
\addcontentsline{toc}{chapter}{结论}

互联网具有快速传播和广泛覆盖的特性,对互联网舆情进行有效监控是必不可少的。
在一个聚焦于新闻、博客和论坛的多通道爬虫系统中,
海量异构、持续变化的特点,给大范围舆情监控带来困难,
迫切需要一种高度自动化的Web信息抽取方式,以降低系统扩展和维护的成本。

本文的主要工作成果如下:

(1)针对新闻、博客这类正文集中的网站,提出了一种基于有效字符的Web内容抽取方法
CEVC(Content Extraction via Valid Characters)。
该方法主要基于这样的观察,网页中不属于链接并包含停止词的文本,更有可能是主要内容。
定义这样的字符为有效字符,根据它们在DOM树中的分布,逐级确定正文区域,并最终提取正文。
与之前的算法CETR(基于文本标签比)、CETD(基于文本密度)
和CEPR(基于文本标签路径比)进行了比较。
实验结果表明,CEVC算法在各项评价指标上都优于CETR和CEPR,
虽然抽取性能和CETD相当,但在预处理阶段依赖更小,适用性更强。

(2)针对论坛网站,提出了一种论坛帖子抽取算法PEAN(Post Extraction via Anchor Nodes)。
该方法利用论坛帖子中普遍存在的发帖时间信息作为锚节点,
根据它们在DOM树中的分布情况,定位论坛帖子集中的区域,并结合树匹配算法,
在候选子树中过滤噪声,最终抽取出论坛帖子。
实验结果表明,PEAN相比于MiBAT在召回率指标上有大幅度提升,
总体F\textsubscript{1}指标也优于MiBAT。

(3)为了验证本文提出的信息抽取算法的实际效果,
根据实际需求设计并实现了一个针对新闻的应用实例——Web新闻聚合系统。
介绍了系统架构和工作流程,并对系统模块和关键技术做了详细阐述,
最后对系统运行效果进行评估。
系统从RSS、元搜索和一般新闻网站三个信息来源采集新闻,
通过模板无关的抽取技术整合了各个渠道的差异,使人工成本限制在新闻列表解析中。

为了进一步提高多通道爬虫系统的自动化采集能力,在以下几个方面还值得继续深入研究:

(1)研究新闻、博客采集源的自动扩展,进一步减少人工参与。

(2)在抽取论坛帖子的基础上,进一步研究作者、发帖内容等元信息的抽取。

(3)研究通用论坛的爬行策略,从论坛中自动解析板块入口和帖子入口。
