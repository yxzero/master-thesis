
\newcommand{\ctitle}{面向多通道爬虫的Web内容提取技术研究}
\newcommand{\cdegree}{工学硕士}
\newcommand{\csubject}{计算机科学与技术}
\newcommand{\caffiliate}{计算机科学与技术学院}
\newcommand{\cauthor}{马雪阳}
\newcommand{\csupervisor}{张宏莉~教授}
\newcommand{\cdate}{2016 年~6 月}

\newcommand{\etitle}
{RESEARCH ON MULTI-CHANNEL CRAWLER ORIENTED WEB CONTENT EXTRACTION TECHNIQUES}
\newcommand{\edegree}{Master of Engineering}
\newcommand{\esubject}{Computer Science and Technology}
\newcommand{\eaffiliate}{School of Computer Science and Technology}
\newcommand{\eauthor}{Xueyang Ma}
\newcommand{\esupervisor}{Prof. Hongli Zhang}
\newcommand{\edate}{June, 2016}

% 工业技术 > 自动化技术、计算机技术 > 计算技术、计算机技术 > 计算机的应用 > 在其他方面的应用
\newcommand{\classifiedindex}{TP399}
% Electrical engineering
\newcommand{\udc}{621.3}
\newcommand{\confidentiality}{公开}

\newcommand{\cabstract}{
随着~Internet~的迅猛发展,网络已经成为一个信息发布和消费的巨大平台。
由于互联网具有快速传播和广泛覆盖的特性,对互联网舆情进行有效监控是必不可少的。
在一个聚焦于新闻、博客和论坛(它们都是很有代表性的信息传播渠道)的多通道爬虫系统中,
我们面临如下挑战:~1~)大量网站需要监控;~2~)网站有不同的结构和布局;~3~)网站会不定期改版。
这些挑战促使我们提出高度自动化的Web内容提取技术,以减少系统的扩展和维护成本。

对于新闻、博客这种正文密集的网站,提出了一个模板无关的基于有效字符的内容提取算法
~CEVC (Content Extraction via Valid Characters)~。
为了验证该方法,从知名的中文、英文新闻和博客网站上任意爬取了部分网页,构成测试数据集进行实验。
实验结果表明~CEVC~能达到平均~$96\%$~的~F\textsubscript{1}-measure~,效果优于之前的算法
~CETR~和~CETD~。

对于典型的论坛网站,利用帖子中普遍存在的发帖时间信息,提出了一个论坛帖子提取算法
~PEAN (Post Extraction via Anchor Nodes)~。
为了和同样利用发帖时间信息的帖子提取算法~MiBAT~比较效果,我们从知名的中文论坛网站上采集网页
进行实验。实验结果表明~PEAN~能达到平均~$94.7\%$~的~F\textsubscript{1}-measure~,
效果优于~MiBAT~。

为了验证本文提出的内容提取算法的实际效果,设计了一个并行多通道爬虫框架,并在此基础上实现了
一个针对新闻网站的爬虫。由于使用了模板无关的内容提取算法,该爬虫能够在没有人工辅助的情况下爬取
新的网站,大大减少了系统扩展的成本。
实际系统的运行情况表明,模板无关的内容提取算法对多通道爬虫系统具有实际意义。
}

\newcommand{\ckeywords}{
多通道;内容提取;模板无关的
}

\newcommand{\eabstract}{
With the explosive growth of the Internet, the Web has become a large platform
for information publishing and consuming. It is essential to supervise the Internet
public opinion effectively due to the rapid dissemination and extensive coverage.
In a multi-channel crawler system which focuses on news, blog and forum, which are
all representative information channels, we face the following challenges:
1) enormous websites should be monitored;
2) websites have different structures and various layouts;
3) websites will change occasionally.
These challenges motivated us to propose highly automated Web content extraction
techniques to reduce the cost for system expansion and maintenance.

For information-intensive websites like Web news and blog, we propose a template
independent content extraction approach based on valid characters (CEVC).
To validate the approach, we conduct experiments by using onling news and blog
files arbitrarily crawled from well-known Chinese and English websites.
Experimental result shows that our method achieves $96\%$ F\textsubscript{1}-measure
on average and outperforms previous methods CETR and CETD.

For typical forum websites, we utilize the ubiquitous date information in forum
posts and propose a forum post extraction method (MiBAN). To compare the effectiveness
with MiBAT, which also uses the date information to extract posts, we conduct
experiments on various Chinese forums. Experimental result shows that our method
achieves $94.7\%$ F\textsubscript{1}-measure on average and outperforms MiBAT.

In order to verify the practicality of the methods, we design a framework for
parallel multi-channel crawler system and implement a Web news crawler based on it.
With the template independent content extraction approach, the crawler has the
ability to crawl new websites without human efforts. The result shows that our
method has practical value on multi-channel crawler system.  
}

\newcommand{\ekeywords}{
multi-channel, content extraction, template independent
}

\makecover
\clearpage 
