
\newcommand{\ctitle}{多元话题竞争传播模型}
\newcommand{\cdegree}{工学硕士}
\newcommand{\csubject}{计算机科学与技术}
\newcommand{\caffiliate}{计算机科学与技术学院}
\newcommand{\cauthor}{于璇}
\newcommand{\csupervisor}{张宏莉~教授}
\newcommand{\cdate}{2017 年~6 月}

\newcommand{\etitle}
{Multi-topic Competition Propagation Model}
\newcommand{\edegree}{Master of Engineering}
\newcommand{\esubject}{Computer Science and Technology}
\newcommand{\eaffiliate}{School of Computer Science and Technology}
\newcommand{\eauthor}{Xuan Yu}
\newcommand{\esupervisor}{Prof. Hongli Zhang}
\newcommand{\edate}{June, 2017}

% 工业技术 > 自动化技术、计算机技术 > 计算技术、计算机技术 > 计算机的应用 > 在其他方面的应用
\newcommand{\classifiedindex}{TP399}
% Electrical engineering
\newcommand{\udc}{621.3}
\newcommand{\confidentiality}{公开}

\newcommand{\cabstract}{
目前,随着在线社交网络的日益普及,社交网络在人们生活中扮演一个非常重要的角色,已经聚集了数亿用户。挖掘社交网络中有价值的信息,利用社交网络信息传播的规律进行热度预测,有助于用户获得需要的信息,企业广告精准投放,以及政府进行有效的舆情调控等。当前已经有很学者致力于发现信息扩散规律,在信息传播领域也有很多重要成果。然而,许多研究主要集中在单条信息是如何在网络中传播,对于多元信息传播分析中关于相互竞争干扰扩散传播模型的研究还处于初级阶段,相关研究工作不多。本文主要研究信息传播中类别的竞争合作关系,提出了一种基于竞争关系矩阵的预测模型,在预测单个用户转发概率以及热度预测任务中,相对于传统模型,各指标得到明显提升。

当前已经有部分传播模型考虑到多条信息同时在网络中传播,但是,他们往往忽略用户和信息之间的关系。同时,以往大多数研究使用隐含特征建模,导致很难结合现实,对信息传播中的规律进行解释。本文使用探针用户的方法,加入用户与信息相互作用矩阵,使预测单个用户转发任务的准确率以及召回率得到提升。

而传统的热度预测模型通常分为两种:1)使用单条信息的早期热度,发布时间等特征进行建模;2)对单条信息每个时段的热度,使用时间序列模型建模预测。这两种方法的通病是,没有考虑到多条新闻信息的相互影响,因此效果没有很好。我们在传统的预测模型基础上加入类别之间的相互作用关系,同时也加入用户与类别相互作用关系,最终得到反应类别之间相互作用关系矩阵,不仅在一定程度上解释信息传播的规律,并有效的改进了预测结果,使得平方损失误差降低35%,平均相对百分误差下降30%。

}

\newcommand{\ckeywords}{
信息传播;竞争关系;热度预测;新闻
}

\newcommand{\eabstract}{
At present, with the growing popularity of online social networks, social networks play a very important role in people's lives, has gathered hundreds of millions of users. Excavate the valuable information in social networks, for the heat prediction using the social network information dissemination rules, helps the user to obtain needed information, advertising accurate delivery, and for effective government regulation of public opinion. Many scholars have devoted to discover the law of information diffusion, and many important achievements have also been made in the field of information dissemination. However, many studies focused on single how information is spread in the network, for the multivariate analysis of dissemination of information about the mutual interference competition diffusion model is still in the primary stage, the related research work is not much. This paper mainly studies the categories of competition and cooperation in information communication, proposes a prediction model of competition based on relation matrix, in the prediction of a single user forwarding probability and heat prediction task, compared with the traditional model, the index has improved significantly. 

At present, partial propagation model takes into account that many messages are transmitted in the network at the same time, but they often ignore the relationship between users and information. At the same time, most of the previous studies use implicit feature modeling, which makes it difficult to combine reality with the interpretation of the law of information dissemination. In this paper, the method of probe users is added to the interaction matrix between users and information, so that the accuracy and recall rate are improved.

The heat of the traditional models are generally divided into two types: 1) early heat using a single information, release time feature modeling; 2) for a single message each time the heat, predictive modeling time series model. The common problem of these two methods is that they do not take into account the interaction of many news information, so the effect is not very good. We join the interaction relationship between categories based on traditional forecasting model, while also adding users and categories of interaction, finally get the response categories, the interaction between relation matrix, not only a part of the explanation for the dissemination of information on the law, and effectively improves the prediction results, the root mean squared error is reduced by 35\%, the mean absolute percentage error decreased by 30\%.

}

\newcommand{\ekeywords}{
information diffusion, competition interaction, hotness prediction, news 
}

\makecover
\clearpage
