
\newcommand{\ctitle}{面向多通道爬虫的Web信息抽取技术研究}
\newcommand{\cdegree}{工学硕士}
\newcommand{\csubject}{计算机科学与技术}
\newcommand{\caffiliate}{计算机科学与技术学院}
\newcommand{\cauthor}{马雪阳}
\newcommand{\csupervisor}{张宏莉~教授}
\newcommand{\cdate}{2016 年~6 月}

\newcommand{\etitle}
{RESEARCH ON WEB INFORMATION EXTRACTION TECHNIQUES FOR MULTI-CHANNEL CRAWLER}
\newcommand{\edegree}{Master of Engineering}
\newcommand{\esubject}{Computer Science and Technology}
\newcommand{\eaffiliate}{School of Computer Science and Technology}
\newcommand{\eauthor}{Xueyang Ma}
\newcommand{\esupervisor}{Prof. Hongli Zhang}
\newcommand{\edate}{June, 2016}

% 工业技术 > 自动化技术、计算机技术 > 计算技术、计算机技术 > 计算机的应用 > 在其他方面的应用
\newcommand{\classifiedindex}{TP399}
% Electrical engineering
\newcommand{\udc}{621.3}
\newcommand{\confidentiality}{公开}

\newcommand{\cabstract}{
随着Internet的迅猛发展,网络已经成为一个信息发布和消费的巨大平台。
互联网具有快速传播和广泛覆盖的特性,对互联网舆情进行有效监控是必不可少的。
由于网页固有的半结构性以及大量存在的与主题无关的噪声,研究如何从Web中抽取人们所需要的信息
变得越来越重要。
在一个聚焦于新闻、博客和论坛(它们都是很有代表性的信息传播渠道)的多通道爬虫系统中,
我们面临如下挑战:
1)大量网站需要监控;
2)网站有不同的结构和布局;
3)网站会不定期改版。
这些挑战促使我们提出高度自动化的Web信息抽取技术,以减少系统的扩展和维护成本。

对于新闻、博客这种正文密集的网站,提出了一个模板无关的基于有效字符的内容抽取算法
CEVC(Content Extraction via Valid Characters)。
为了验证该方法,从知名的中文新闻和博客网站上任意爬取了部分网页,构成测试数据集进行实验。
实验结果表明CEVC能达到平均95.8\%的F\textsubscript{1}-measure,
效果优于之前的算法CETR和CEPR,虽然抽取性能和CETD相当,
但在预处理阶段依赖更小,适用性更强。

对于典型的论坛网站,利用帖子中普遍存在的发帖时间信息,提出了一个论坛帖子抽取算法
PEAN(Post Extraction via Anchor Nodes)。
为了和同样利用发帖时间信息的帖子抽取算法MiBAT比较效果,
我们从知名的中文论坛网站上采集网页进行实验。
实验结果表明PEAN相比于MiBAT在召回率指标上有大幅度提升,
平均94.7\%的F\textsubscript{1}-measure也优于MiBAT。

为了验证本文提出的信息抽取算法的实际效果,
我们针对实际需求设计并实现了一个Web新闻采集系统。
由于使用了模板无关的内容抽取算法,该爬虫能够在较少人工辅助的情况下爬取新的网站,
大大减少了系统扩展和维护的成本。
实际系统的运行情况表明,模板无关的内容抽取算法对多通道爬虫系统具有实际意义。
}

\newcommand{\ckeywords}{
多通道;爬虫;信息抽取;模板无关
}

\newcommand{\eabstract}{
With the explosive growth of the Internet, the Web has become a large platform
for information publishing and consuming. It is essential to supervise the Internet
public opinion effectively due to the rapid dissemination and extensive coverage.
As the inherent semi-structured characteristics and large part of topic irrelevant
noises, effectively extracting main content and filtering these noises is
necessary and challenging.
In a multi-channel crawler system which focuses on news, blog and forum, which are
all representative information channels, we face the following challenges:
1) enormous websites should be monitored;
2) websites have different structures and various layouts;
3) websites will change occasionally.
These challenges motivated us to propose highly automated Web information extraction
techniques to reduce the cost for system expansion and maintenance.

For information-intensive websites like Web news and blog, we propose a template
independent content extraction approach based on valid characters (CEVC).
To validate the approach, we conduct experiments by using onling news and blog
files arbitrarily crawled from well-known Chinese news and blog websites.
Experimental result shows that our method achieves 95.8\% F\textsubscript{1}-measure
on average and outperforms previous methods CETR and CEPR.
Although CEVC has almost equivalent extraction performance as CETD,
CEVC has less dependence in the pre-processing stage thus more applicable.

For typical forum websites, we utilize the ubiquitous date information in forum
posts and propose a forum post extraction method (PEAN). To compare the effectiveness
with MiBAT, which also uses the date information to extract posts, we conduct
experiments on various Chinese forums. Experimental result shows that our method
achieves much higher recall than MiBAT, and the F\textsubscript{1}-measure 
of 94.7\% also outperforms MiBAT.

In order to verify the practicality of the methods, we design a framework for
parallel multi-channel crawler system and implement a Web news crawler based on it.
With the template independent content extraction approach, the crawler has the
ability to crawl new websites with little human efforts. The result shows that
template independent content extraction methods have practical value on
on multi-channel crawler system.
}

\newcommand{\ekeywords}{
multi-channel, crawler, information extraction, template independent
}

\makecover
\clearpage
